\chapter{Solutions for Chapter 2}

\ex{2.1}
In order to solve this problem, many assumptions must be made.
Different people may assume slightly different values for parameters.
This is OK.
What is important is making good assumptions and checking our conclusions to make sure they are reasonable.

To solve for the current in the LED, let us assume we know the LED is red, so it follows the red LED curve from Figure 2.8 in the book.
Let us also assume the transistor is acting like a closed switch, so the collector voltage of Q1 is close to \SI{0}{\V}.
Let us also assume the LED is ON, so it's voltage is approximately $V_\text{LED} = \SI{2}{V}$. From the preceding assumptions, we can calculate that the LED current is
\[I_\text{LED} = \frac{\SI{3.3}{\V} - \SI{2}{\V}}{\SI{330}{\ohm}} = \frac{\SI{1.3}{\V}}{\SI{330}{\ohm}} \approx \SI{3.94}{\mA}\]
If we use Figure 2.8 (from the textbook) to check our numbers, we see that a current of \SI{3.94}{\mA} roughly correlates to an LED voltage of $V_\text{LED} = \SI{1.7}{V}$.  We will run the same calculation again to reduce our error.
\[I^{*}_\text{LED} = \frac{\SI{3.3}{\V} - \SI{1.7}{\V}}{\SI{330}{\ohm}} = \frac{\SI{1.6}{\V}}{\SI{330}{\ohm}} \approx \mans{\SI{4.85}{\mA}}\]

In order to determine the minimum current gain required from our transistor, we must calculate the base current.
Let us assume we know the base-emitter voltage $V_\text{BE} = \SI{0.6}{\V}$.
Therefore
\[I_\text{B} = \frac{\SI{3.3}{\V} - \SI{0.6}{\V}}{\SI{10}{\kohm}} = \SI{270}{\uA}\]
So the minimum current gain must be
\[\beta_\text{min} = \frac{I^{*}_\text{LED}}{I_\text{B}} \approx \frac{\SI{4.85}{\mA}}{\SI{270}{\uA}} \approx \mans{18.0}\]

\ex{2.2}
Before the pulse, $V_\text{in} = V^{Q_1}_\text{B} = \SI{0}{\V}$, therefore transistor $Q_1$ is turned off, so $V^{Q_1}_\text{C} = \SI{5}{\V}$.
Transistor $Q_2$ is turned on through $R_3$ and has $V^{Q_2}_\text{B} = \SI{0.6}{\V}$,
$I^{Q_2}_\text{B} = \frac{\SI{4.4}{\V}}{\SI{10}{\kohm}} = \SI{0.44}{\mA}$.
Since $R_2$ is only $\SI{10}{\kohm}$, $Q_2$ is saturated and $V_\text{out} = \SI{0}{\V}$.
The potential difference accross the capacitor is \[\Delta V_{C_1} = V^{Q_1}_\text{C} - V^{Q_2}_\text{B} = \SI{4.4}{\V}\]

Right after the pulse, $Q_1$ turns on, having $V^{Q_1}_\text{B} = \SI{0.6}{\V}$ and $I^{Q_1}_\text{B} = \frac{\SI{4.4}{\V}}{\SI{10}{\kohm}} = \SI{0.44}{\mA}$.
Since $R_2$ is only $\SI{1}{\kohm}$, $Q_1$ saturates and has $V^{Q_1}_\text{C} = \SI{0}{\V}$.
To keep the potential difference over $C_1$ the same, $V^{Q_2}_\text{B} = \SI{-4.4}{\V}$. This causes $Q_2$ to instantaneously turn off, resulting in $V_\text{out} = \SI{5}{\V}$.

While $V_\text{in}$ is high, $C_1$ starts charging through $R_3$ untill $V^{Q_2}_\text{B}=\SI{0.6}{\V}$, after which $Q_2$ turns on again.
Before that $C_1$ charges as if $Q_2$ was not connected.
\[V^{Q_1}_\text{B}(t) = V_\text{CC}-[V_\text{CC}-V^{Q_1}_\text{B}(0)] e^{t/R_3C_1} = \SI{5}{\V}-\SI{9.4}{\V} e^{t/R_3C_1}\]
where $t=0$ is the start of the pulse.
The pulse stops when $Q_2$ turns on so
\[V^{Q_1}_\text{B}{\tau_\text{pulse}}=\SI{0.6}{\V} = \SI{5}{\V}-\SI{9.4}{V} e^{\tau_\text{pulse}/R_3C_1}\]
Solving for $\tau_\text{pulse}$ gives
\[\tau_\text{pulse} = \mans{0.76 R_3 C_1} = \mans{\SI{76}{\us}}\]

\ex{2.3}
During the pulse $V_\text{out}$ is high, turning $Q_3$ on: $V^{Q_3}_\text{B} = \SI{0.6}{\V}$.
$R_4$ and $R_5$ form a voltage divider such that the output is
\[V_\text{out} = (V_\text{CC}-V^{Q_3}_\text{B})\frac{R_5}{R_5+R_4}+V^{Q_3}_\text{B} = \SI{4.8}{\V}\]
The output is therefore reduced by $\mans{\SI{0.2}{\V}}$. A high value $R_5$ keeps this drop low.

For $Q_3$ to stay saturated it has to be able to drain a collector current of at least $\frac{V_\text{CC}}{R_2}=\SI{5}{\mA}$, such that we must have $I^{Q_3}_\text{B} \geq \frac{\SI{5}{\mA}}{\beta}$. So we have $I^{Q_3}_\text{B} = \frac{V_\text{out}-V^{Q_3}_\text{B}}{R_5} = \frac{\SI{4.8}{\V}-\SI{0.6}{\V}}{\SI{20}{\kohm}} = \SI{0.21}{\mA} \geq \frac{\SI{5}{\mA}}{\beta}$.
The resulting minimum $\beta$ is therefore
\[\mans{\beta_\text{min} \geq 24}\]
In practice you probably want a larger $\beta$ to account for the additional voltage drop and hence $Q_3$ base current drop that a load will cause.


\todoex{2.4}

\todoex{2.5}

\todoex{2.6}

\todoex{2.7}

\todoex{2.8}

\todoex{2.9}

\todoex{2.10}

\todoex{2.11}

\todoex{2.12}

\todoex{2.13}

\todoex{2.14}

\todoex{2.15}

\todoex{2.16}

\todoex{2.17}

\todoex{2.18}

\todoex{2.19}

\todoex{2.20}

\todoex{2.21}

\todoex{2.22}

\todoex{2.23}

\todoex{2.24}

\todoex{2.25}

\todoex{2.26}

\todoex{2.27}

\todoex{2.28}

\todoex{2.29}

\todoex{2.30}  % the last Exercise in chapter 2.

% Here ends Chapter 2.
