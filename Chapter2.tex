\chapter{Solutions for Chapter 2}

\ex{2.1}
In order to solve this problem, many assumptions must be made.
Different people may assume slightly different values for parameters.
This is OK.
What is important is making good assumptions and checking our conclusions to make sure they are reasonable.

To solve for the current in the LED, let us assume we know the LED is red, so it follows the red LED curve from Figure 2.8 in the book.
Let us also assume the transistor is acting like a closed switch, so the collector voltage of Q1 is close to \SI{0}{\V}.
Let us also assume the LED is ON, so it's voltage is approximately $V_\text{LED} = \SI{2}{V}$. From the preceding assumptions, we can calculate that the LED current is
\[I_\text{LED} = \frac{\SI{3.3}{\V} - \SI{2}{\V}}{\SI{330}{\ohm}} = \frac{\SI{1.3}{\V}}{\SI{330}{\ohm}} \approx \SI{3.94}{\mA}\]
If we use Figure 2.8 (from the textbook) to check our numbers, we see that a current of \SI{3.94}{\mA} roughly correlates to an LED voltage of $V_\text{LED} = \SI{1.7}{V}$.  We will run the same calculation again to reduce our error.
\[I^{*}_\text{LED} = \frac{\SI{3.3}{\V} - \SI{1.7}{\V}}{\SI{330}{\ohm}} = \frac{\SI{1.6}{\V}}{\SI{330}{\ohm}} \approx \mans{\SI{4.85}{\mA}}\]

In order to determine the minimum current gain required from our transistor, we must calculate the base current.
Let us assume we know the base-emitter voltage $V_\text{BE} = \SI{0.6}{\V}$.
Therefore
\[I_\text{B} = \frac{\SI{3.3}{\V} - \SI{0.6}{\V}}{\SI{10}{\kohm}} = \SI{270}{\uA}\]
So the minimum current gain must be
\[\beta_\text{min} = \frac{I^{*}_\text{LED}}{I_\text{B}} \approx \frac{\SI{4.85}{\mA}}{\SI{270}{\uA}} \approx \mans{18.0}\]


\ex{2.2}
When $Q_1$ goes is in saturation, the base voltage of $Q_2$ equals the opposite of the voltage on the capacitor $C_1$ at $t = \SI{0}{\second}$, $V_0 = \SI{4.4}{\volt}$ and $Q_2$ is then cutoff. $V_\text{out}$ will be equal to \SI{5}{\volt} until $Q_2$ is brought in saturation again. This happens when its base voltage gets higher or equal to the $Q_2$ threshold voltage (\SI{0.6}{\volt}). As soon as $Q_1$ is brought in saturation, $C_1$ starts to discharge into the resistor $R_3$ and the equivalent circuit, valid until $Q_2$ is cutoff, is then:
\begin{circuit}{fig:2.2.1}{Equivalent $C_1$ discharging circuit.}
    (0,0) node[ground] {}
    to[C, l=$C_1$,v=$V_\text{C}$] ++(0,2)
    to[R, l=$R_3$] ++(2.5,0)
    node[above] {+\SI{5}{\volt}} node[ocirc] {}
\end{circuit}
The time evolution of the voltage across the capacitor $C_1$ is given by:
\[V_\text{C}(t)=\left(V_0-V_\infty\right)e^{-\frac{t}{R_3C_1}}+V_\infty\]
where $V_\infty$ is the steady-state voltage on the capacitor $C_1$ end equals \SI{-5}{\volt}. Given the considerations above, we have that $V_C(t=T_\text{pulse})=\SI{-0.6}{\volt}$. Solving for $t$ gives:
\[T_\text{pulse}=-R_3C_1\ln\left(\frac{\SI{-0.6}{\volt}-V_\infty}{{V_0-V_\infty}}\right)=\mans{0.76R_3C_1=\SI{76}{\micro\second}}\]


\ex{2.3}
The output voltage is now influenced by $R_5$ that goes in series with $R_4$, and by the $V_\text{BE}$ of $Q_3$ which is equal to \SI{0.6}{\volt} when the transistor is in saturation. Therefore:
\[V_\text{out}=\frac{R_5}{R_5+R_4}\left(\SI{5}{\volt}-\SI{0.6}{\volt}\right)+\SI{0.6}{\volt}=\mans{\SI{4.79}{\volt}}\]

The minimum value of $\beta$ of $Q_3$ can be obtained looking at the maximum value of the current flowing through the collector of $Q_3$, $I_\text{c}^{\text{Q}_3}$. As soon as $Q_1$ goes in saturation, the capacitor $C_1$ starts to discharge and its current is given by $C_1dV_\text{C}/dt$. With reference to the variables introduced in the previous exercise (2.2):
\[I_\text{c}^{\text{Q}_3}(t)=\frac{\SI{5}{\volt}}{R_2}-I_{\text{C}_1}(t)=\frac{\SI{5}{\volt}}{R_2}+C_1\frac{1}{R_3C_1}\left(V_0-V_\infty\right)e^{-\frac{t}{R_3C_1}}\]
Therefore:
\[\beta_\text{min}=\frac{I_\text{c}^{\text{Q}_3}(t)|_\text{max}}{I_\text{b}^{\text{Q}_3}}=\frac{I_\text{c}^{\text{Q}_3}(t=\SI{0}{\second})}{I_\text{b}^{\text{Q}_3}}=\mans{27}\]


\ex{2.4}
\begin{circuit}{fig:2.4.1}{Emitter follower circuit used for computing the output resistance}
    (0,0) node[npn] (Q) {Q}
    (Q.B) to[short, i_<=$i_\text{b}$] ++(-1,0) -- ++(0,-.5) 
    to[R, l=$R_\text{S}$, v=$v_\text{Rs}$] ++(0,-2)
    node[ground] {}
    (Q.C) to[short, i_<=$i_\text{c}$] ++(0,.5) node[ground, rotate=180] {}
    (Q.E) to[short, i_>=$i_\text{e}$] ++(0,-.5) coordinate(RE)
    to[R, l=$R_\text{E}$,i=$i_\text{Re}$] ++(0,-2)
    node[ground] {}
    (RE) -- ++(3,0)
    to[I, l=$i_\text{out}$, invert, v_>=$v_\text{out}$] ++(0,-2)
    node[ground] {}
\end{circuit}

Applying the KCL on the Q transistor:
\[i_\text{e}=i_\text{b}+i_\text{c}=i_\text{b}\left(\beta+1\right)\]
The current flowing through the emitter resistor $R_\text{E}$ is equal to:
\[i_\text{Re}=i_\text{e}+i_\text{out}=i_\text{b}\left(\beta+1\right)+i_\text{out}\]
Since for the emitter follower $v_\text{Rs}=v_\text{out}$:
\[\left[i_\text{b}\left(\beta+1\right)+i_\text{out}\right]R_\text{E}=v_\text{out}\]
Since:
\[i_\text{b}=-\frac{v_\text{Rs}}{R_\text{S}}=-\frac{v_\text{out}}{R_\text{S}}\]
we can write:
\[\left[-\frac{v_\text{out}}{R_\text{S}}\left(\beta+1\right)+i_\text{out}\right]R_\text{E}=v_\text{out}\]
Therefore:
\[R_\text{out}=\frac{v_\text{out}}{i_\text{out}}=\frac{R_\text{E}R_\text{S}}{R_\text{S}+\left(\beta+1\right)R_\text{E}}\]
If $R_\text{E}>>R_\text{S}/(\beta+1)$:
\[\mans{R_\text{out}\approx\frac{R_\text{S}}{\left(\beta+1\right)}}\]


\ex{2.5}
\begin{circuit}{fig:2.5.1}{Small signal circuit}
    (0,0) node[npn] (Q) {Q}
    (Q.B) to[short, i_<=$i_\text{b}$] ++(-1,0) coordinate(RP)
    -- ++(0,-.5) 
    to[R, l=$R_2$, v=$v_\text{in}$] ++(0,-2)
    node[ground] {}
    (RP) -- ++(0,.5)
    to[R, l=$R_1$] ++(0,2)
    -- ++(-3,0)
    to[sV=\SI{15}{\volt}] ++(0,-2)
    node[ground] {}
    (Q.C) to[short, i_<=$i_\text{c}$] ++(0,.5) node[ground, rotate=180] {}
    (Q.E) to[short, i_>=$i_\text{e}$] ++(0,-.5) coordinate(RE)
    to[R, l=$R_\text{L}$] ++(0,-2)
    node[ground] {}
\end{circuit}

In order to achieve a maximum voltage change of \SI{5}{\percent} for a maximum current to the load ($R_\text{L}$) equal to \SI{25}{\milli\ampere}, we can make reference to the equivalent circuit of Figure \ref{fig:2.5.2}:
\begin{circuit}{fig:2.5.2}{Output equivalent circuit}
    (0,0) node[ground] {}
    to[sV,v_<=$v_\text{out}$, invert] ++(0,2)
    to[short, i=$i_\text{out}$] ++(.5,0)
    to[R,l=$R_\text{out}$] ++(2,0)
    -- ++(.5,0)
    to[R,l=$R_\text{L}$, v=$v_\text{L}$] ++(0,-2)
    node[ground] {}
\end{circuit}
obtaining:
\[\left.\frac{v_\text{out}-v_\text{L}}{v_\text{out}}\right\rvert_{i_\text{out}=\SI{25}{\milli\ampere}}=0.05\]
Since
\[v_\text{out}-R_\text{out}\,i_\text{out}=v_\text{L}\]
and for an emitter follower $v_\text{out}=v_\text{in}=\SI{5}{\volt}$
we can write:
\[\frac{R_\text{out}\,\SI{25}{\milli\ampere}}{\SI{5}{\volt}}=0.05\]
obtainin the following condition on $R_\text{out}$:
\[R_\text{out}=\frac{0.05\,\SI{5}{\volt}}{\SI{25}{\milli\ampere}}\]
For the emitter follower configuration:
\[R_\text{out}=\frac{R_\text{in}}{\beta+1}\]
and we see from the circuit of Figure \ref{fig:2.5.1} that $R_\text{in}$ is given by the parallel between $R_\text{1}$ and $R_\text{2}$:
\[R_\text{in}=\frac{R_\text{1}\,R_\text{2}}{R_\text{1}+R_\text{2}}\]
In order to achieve $v_\text{in}=\SI{5}{\volt}$, the following condition must be verified for the values of $R_\text{1}$ and $R_\text{2}$:
\[\frac{R_\text{2}}{R_\text{1}+R_\text{2}}=\frac{\SI{5}{\volt}}{\SI{15}{\volt}}\]
Assuming $\beta=100$, we can finally obtain:
\[\mans{R_\text{1}=\SI{30}{\ohm},\,R_\text{2}=\SI{15}{\ohm}}\]


\ex{2.6}
The minimum current flowing through the $R$ resistor has to be at least equal to the maximum current to the load plus the minimum current to the zener:
\[I_\text{min,R}=\frac{\SI{20}{\volt}-\SI{10}{\volt}}{R}\geq\SI{100}{\milli\ampere}+\SI{10}{\milli\ampere}\]
Therefore:
\[R\leq \frac{\SI{10}{\volt}}{\SI{110}{\milli\ampere}}=\mans{\SI{91}{\ohm}}\]
It follows that the maximum power to the zener, selecting $R=\SI{91}{\ohm}$, is equal to
\[P_\text{max,z}=\left(\frac{\SI{25}{\volt}-\SI{10}{\volt}}{\SI{91}{\ohm}}-\SI{0}{\ampere}\right)\SI{10}{\volt}=\mans{\SI{1.65}{\watt}}\]


\ex{2.7}
With reference to figure 2.21 of the book, neglecting the current entering the ase of the transistor $Q$, in order to have at least \SI{10}{\milli\ampere} flowing through the zener, the resistor $R$ should comply with the following condition:
\[\frac{\SI{20}{\volt}-\SI{10}{\volt}}{R}\geq\SI{10}{\milli\ampere}\]
which results in:
\[\mans{R\leq\SI{1}{\kilo\ohm}}\]
In order to avoid the transistor to be saturated, we want the collector-base voltage to be always higher than zero. This translates in:
\[R_\text{C}<R\frac{\SI{10}{\milli\ampere}}{\SI{100}{\milli\ampere}}=\SI{100}{\ohm}\]
Selecting a conservative value of $R_\text{C}$ equal to \tans{\SI{20}{\ohm}}, we can compute the maximum power dissipated by the zener, $P_\text{max,z}$ and the transistor, $P_\text{max,Q}$ as:
\[P_\text{max,z}=\left(\frac{\SI{25}{\volt}-\SI{10}{\volt}}{\SI{1}{\kilo\ohm}}\right)\SI{10}{\volt}=\mans{\SI{0.15}{\watt}}\]
\[P_\text{Q}=\left(\SI{25}{\volt}-\SI{20}{\ohm}I_\text{load}\right)I_\text{load}\]
The maximum power dissipated by the transistor is obtained for a collector current equal to \SI{625}{\milli\ampere} which is higher than the maximum load current. Therefore, in our case, the maximum power dissipated by $Q$ will be obtained for $I_\text{load,max}=\SI{100}{\milli\ampere}$:
\[P_\text{max,Q}=\left(\SI{25}{\volt}-\SI{20}{\ohm}I_\text{load,max}\right)I_\text{load,max}=\mans{\SI{2.3}{\watt}}\]
Comparing the results with those of the previous exercise, we notice that the power disspated by the zener diode significantly decreased but we have an additional power dissipated by the transistor which is higher than the power that the zener diode dissipated in the circuit of the previous exercise. However this power can be decreased by increasing the value of $R_\text{C}$.


\ex{2.8}
In order to keep the emitter voltage $V_\text{E}$ in the half range of the dc supply, considering the quiescent current of \SI{5}{\milli\ampere} we have:
\[V_\text{E}=\frac{\SI{15}{\volt}-(-\SI{15}{\volt})}{2}=\SI{15}{\volt}\]
and the emitter resistor $R_\text{E}$:
\[R_\text{E}=\frac{\SI{15}{\volt}}{\SI{5}{\milli\ampere}}=\mans{\SI{3}{\kilo\ohm}}\]
Since the imput impedance to the transistor, under the assumption of a load resistance much larger than the emitter resistance, is:
\[R_\text{in}=\beta R_\text{E}=\SI{300}{\kilo\ohm}\]
in order to have the \SI{3}{\decibel} point below the lowest frequency of \SI{20}{\hertz}, the capacitor $C_1$ has to be:
\[\frac{1}{R_\text{in}C_1}\leq \SI{20}{\hertz}\]
meaning that:
\[\mans{C_1\geq\SI{0.17}{\micro\farad}}\]


\ex{2.9}
\begin{circuit}{fig:2.9.1}{Current source circuit}
    (0,0) node[ground] {}
    to[V,v_<=$V_\text{S}$, invert] ++(0,2)
    -- ++(1,0)
    to[R, l=$R$] ++(2,0)
    to[short,i=$I_\text{load}$] ++(1,0)
    node[ocirc] {}
    to[open,v=$V_\text{load}$] ++(0,-2)
    node[ocirc] {}
    node[ground] {}
\end{circuit}
We want:
\[\frac{I_\text{load}^\text{max}-I_\text{load}^\text{min}}{I_\text{load}^\text{max}}=\frac{V_\text{S}-\SI{0}{\volt}-(V_\text{S}-\SI{10}{\volt})}{V_\text{S}-\SI{0}{\volt}}=0.01\]
from which it follows:
\[\mans{V_\text{S}=\SI{1}{\kilo\volt}}\]


\ex{2.10}
With reference to Figure \ref{fig:2.9.1}, we assume that $I_\text{load}$ is equal to \SI{10}{\milli\ampere} if $V_\text{load}$ is equal to \SI{0}{\volt}, which means $R_\text{load}$ is equal to \SI{0}{\ohm}. We can therfore calculate $R$ as:
\[R=\frac{V_\text{S}}{\SI{10}{\milli\ampere}}=\SI{100}{\kilo\ohm}\]
In this case we have:
\[\mans{P_\text{load}=\SI{0}{\watt},\,P_\text{R}=I_\text{load}^2R=\SI{10}{\watt}}\]
If $V_\text{load}=\SI{10}{\volt}$, from the condition of the previous exercise, we have:
\[I_\text{load}=(1-0.01)\SI{10}{\milli\ampere}=\SI{9.9}{\milli\ampere}\]
Therefore:
\[\mans{P_\text{load}=\SI{10}{\volt}I_\text{load}=\SI{0.1}{\watt},\,P_\text{R}=I_\text{load}^2R=\SI{9.8}{\watt}}\]

\todoex{2.11}

\todoex{2.12}

\todoex{2.13}

\todoex{2.14}

\todoex{2.15}

\todoex{2.16}

\todoex{2.17}

\todoex{2.18}

\todoex{2.19}

\todoex{2.20}

\todoex{2.21}

\todoex{2.22}

\todoex{2.23}

\todoex{2.24}

\todoex{2.25}

\todoex{2.26}

\todoex{2.27}

\todoex{2.28}

\todoex{2.29}

\todoex{2.30}  % the last Exercise in chapter 2.

% Here ends Chapter 2.
