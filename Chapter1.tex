\documentclass{article}
\usepackage{circuitikz}
\usepackage{amsmath}
\usepackage{enumitem}

\newcommand{\todo}[1]{\textcolor{red}{\textbf{TODO: #1}}}
\newcommand{\ex}[1]{\section*{Exercise #1}}
\newcommand{\mans}[1]{\boxed{\mathbf{#1}}}
\newcommand{\tans}[1]{\framebox{\textbf{#1}}}

% Make easier use of EE order of magnitude notation
\renewcommand{\k}{\ensuremath{\text{k}}} % kilo
\newcommand{\M}{\ensuremath{\text{M}}} % Mega

\newcommand{\V}{\ensuremath{\text{V}}}
\newcommand{\Ohm}{\ensuremath{\Omega}}
\newcommand{\W}{\ensuremath{\text{W}}}

\newcommand{\Th}{\ensuremath{\text{Th}}} % Thevenin
\renewcommand{\in}{\ensuremath{\text{in}}} % In (for V_{In})
\newcommand{\out}{\ensuremath{\text{out}}}

\newenvironment{circuit}
    {\begin{center} 
    \begin{circuitikz}[american]\draw
    }
    {;
     \end{circuitikz}
    \end{center}
    }

% Set first level of list to (a) format
\setlist[enumerate,1]{label=(\alph*)}

% Be sure to use \boxed{<ansswer>} to box the answers and maintain 

\begin{document}

    \ex{1.1}
    \begin{enumerate}
        \item 
        $R = 5\k + 10\k = \mans{15\k\Omega}$

        \item 
        $R = \dfrac{R_1 R_2}{R_1 + R_2} = \dfrac{5\k \cdot 10\k}{5\k + 10\k} = \mans{3.33\k\Omega}$

    \end{enumerate}

    \ex{1.2}
    $P = IV = \left(\dfrac{V}{R}\right)V = \dfrac{(12\V)^2}{1\Ohm} = \mans{144\W}$

    \ex{1.3}
    \todo{Solve this problem}

    \ex{1.4}
    \todo{Solve this problem}

    \ex{1.5}
    Given that $P = \dfrac{V^2}{R}$, we know that the maximum voltage we can achieve is 15V and the smallest resistance we can have across the resistor in question is $1\k\Ohm$. Therefore, the maximum amount of power dissipated can be given by \[P = \frac{V^2}{R} = \frac{(15\V)^2}{1\k\Ohm} = \mans{0.225\W}\]
    This is less than the 1/4W power rating.

    \ex{1.10}
    \begin{enumerate}
        \item 
        With two equal-value resistors, the output voltage is half the input voltage.
        \[V_{out} = \frac{1}{2}V_{in} = \frac{30\V}{2} = \mans{15\V}\]

        \item 
        To treat $R_2$ and $R_{load}$ as a single resistor, combine the two resistors which are in parallel to find that the combined (equivalent) resistance is $5\k\Ohm$. Now, we have a simple voltage divider with a $10\k\Ohm$ resistor in series with the $5\k\Ohm$ equivalent resistor. The output voltage is across this equivalent resistance. The output voltage is given by 
        \[V_{out} = V_{in} \frac{5\k\Ohm}{10\k\Ohm + 5\k\Ohm} = \frac{30\V}{3} = 10\V \]

        \item 
        We can redraw the voltage divider circuit to make the ``port'' clearer.
        \begin{circuit}
            (0,2) to[V=$V_{\in}$] (0,0)
            to[short] (2,0)
            to[R=$R_2$] (2,2)
            to[R=$R_1$](0,2)
            (2,0) to[short, *-o] (3,0)
            (2,2) to[short, *-o] (3,2)
            (3,0) to[open, v_<=$V_\out$] (3,2)
        \end{circuit}

        We can find $V_{\Th}$ by leaving the ports open (open circuit) and measuring $V_\out$, the voltage across $R_2$. This comes out to be half the input voltage when $R_1 = R_2$, so $V_\out = 15\V$. Thus $V_{\Th} = 15\V$.
        
        To find the Th\'evinen resistance, we need to find the short circuit current, $I_{SC}$.
        
        \todo{finish}

        \begin{circuit}
            (0,2) to[V=$V_{\Th}$] (0,0)
            to[short, -o] (3,0)
            (0,2) to[R=$R_{\Th}$, -o] (3,2)
            ;
        \end{circuit}
    \end{enumerate}
    % \begin{circuitikz} \draw 
    %     (0,0) to[battery] (0,4)
    %           to[ammeter] (4,4) 
    %           to[american inductor] (4,0)
    %           to[lamp] (0,0)
    % ;
    % \end{circuitikz}
\end{document}